% --------------------------------------------------------------
% This is all preamble stuff that you don't have to worry about.
% Head down to where it says "Start here"
% --------------------------------------------------------------
 
\documentclass[12pt]{article}
\usepackage{amsgen,amsmath,amstext,amsbsy,amsopn,amssymb,mathabx,amsthm,bm,bbm,romannum,enumitem}
\usepackage[dvips]{graphicx}

\usepackage[pagebackref,bookmarksnumbered]{hyperref}
\usepackage{url}
\hypersetup{
	colorlinks=true,
	linkcolor=red,
	filecolor=magenta,      
	urlcolor=blue,
}

\setcounter{tocdepth}{3}
\usepackage[depth=3]{bookmark}

\usepackage[margin=1in]{geometry}
\renewcommand{\baselinestretch}{1.5}	% Line Stretch

\usepackage[utf8]{inputenc}

%----- theorems -----%

\newtheorem{thm}{Theorem}[section]
\newtheorem{lem}[thm]{Lemma}
\newtheorem{prop}[thm]{Proposition}
\newtheorem{coro}[thm]{Corollary}

\theoremstyle{definition}
\newtheorem{dfn}{Definition}[section]
\newtheorem*{pchln}{Punchline}

\theoremstyle{remark}
\newtheorem*{rmk}{Remark}
\newtheorem*{note}{Note}
\newtheorem{eg}{Example}[section]
\newtheorem{fact}{Fact}[section]
\newtheorem*{hint}{Hint}


%----- bold fonts -----%

\newcommand{\ab}{\mathbf{a}}
\newcommand{\bbb}{\mathbf{b}}
\newcommand{\cbb}{\mathbf{c}}
\newcommand{\db}{\mathbf{d}}
\newcommand{\eb}{\mathbf{e}}
\newcommand{\fb}{\mathbf{f}}
\newcommand{\gb}{\mathbf{g}}
\newcommand{\hb}{\mathbf{h}}
\newcommand{\ib}{\mathbf{i}}
\newcommand{\jb}{\mathbf{j}}
\newcommand{\kb}{\mathbf{k}}
\newcommand{\lb}{\mathbf{l}}
\newcommand{\mb}{\mathbf{m}}
\newcommand{\nbb}{\mathbf{n}}
\newcommand{\ob}{\mathbf{o}}
\newcommand{\pb}{\mathbf{p}}
\newcommand{\qb}{\mathbf{q}}
\newcommand{\rb}{\mathbf{r}}
\newcommand{\sbb}{\mathbf{s}}
\newcommand{\tb}{\mathbf{t}}
\newcommand{\ub}{\mathbf{u}}
\newcommand{\vb}{\mathbf{v}}
\newcommand{\wb}{\mathbf{w}}
\newcommand{\xb}{\mathbf{x}}
\newcommand{\yb}{\mathbf{y}}
\newcommand{\zb}{\mathbf{z}}

% denote vectors
\newcommand{\ba}{\bm{a}}
\newcommand{\bb}{\bm{b}}
\newcommand{\bc}{\bm{c}}
\newcommand{\bd}{\bm{d}}
\newcommand{\be}{\bm{e}}
\newcommand{\bbf}{\bm{f}}
\newcommand{\bg}{\bm{g}}
\newcommand{\bh}{\bm{h}}
\newcommand{\bi}{\bmf{i}}
\newcommand{\bj}{\bm{j}}
\newcommand{\bk}{\bm{k}}
\newcommand{\bl}{\bm{l}}
\newcommand{\bbm}{\bm{m}}
\newcommand{\bn}{\bm{n}}
\newcommand{\bo}{\bm{o}}
\newcommand{\bp}{\bm{p}}
\newcommand{\bq}{\bm{q}}
\newcommand{\br}{\bm{r}}
\newcommand{\bs}{\bm{s}}
\newcommand{\bt}{\bm{t}}
\newcommand{\bu}{\bm{u}}
\newcommand{\bv}{\bm{v}}
\newcommand{\bw}{\bm{w}}
\newcommand{\bx}{\bm{x}}
\newcommand{\by}{\bm{y}}
\newcommand{\bz}{\bm{z}}

% denote random matrices
\newcommand{\Ab}{\mathbf{A}}
\newcommand{\Bb}{\mathbf{B}}
\newcommand{\Cb}{\mathbf{C}}
\newcommand{\Db}{\mathbf{D}}
\newcommand{\Eb}{\mathbf{E}}
\newcommand{\Fb}{\mathbf{F}}
\newcommand{\Gb}{\mathbf{G}}
\newcommand{\Hb}{\mathbf{H}}
\newcommand{\Ib}{\mathbf{I}}
\newcommand{\Jb}{\mathbf{J}}
\newcommand{\Kb}{\mathbf{K}}
\newcommand{\Lb}{\mathbf{L}}
\newcommand{\Mb}{\mathbf{M}}
\newcommand{\Nb}{\mathbf{N}}
\newcommand{\Ob}{\mathbf{O}}
\newcommand{\Pb}{\mathbf{P}}
\newcommand{\Qb}{\mathbf{Q}}
\newcommand{\Rb}{\mathbf{R}}
\newcommand{\Sbb}{\mathbf{S}}
\newcommand{\Tb}{\mathbf{T}}
\newcommand{\Ub}{\mathbf{U}}
\newcommand{\Vb}{\mathbf{V}}
\newcommand{\Wb}{\mathbf{W}}
\newcommand{\Xb}{\mathbf{X}}
\newcommand{\Yb}{\mathbf{Y}}
\newcommand{\Zb}{\mathbf{Z}}

% denote random vectors
\newcommand{\bA}{\bm{A}}
\newcommand{\bB}{\bm{B}}
\newcommand{\bC}{\bm{C}}
\newcommand{\bD}{\bm{D}}
\newcommand{\bE}{\bm{E}}
\newcommand{\bF}{\bm{F}}
\newcommand{\bG}{\bm{G}}
\newcommand{\bH}{\bm{H}}
\newcommand{\bI}{\bm{I}}
\newcommand{\bJ}{\bm{J}}
\newcommand{\bK}{\bm{K}}
\newcommand{\bL}{\bm{L}}
\newcommand{\bM}{\bm{M}}
\newcommand{\bN}{\bm{N}}
\newcommand{\bO}{\bm{O}}
\newcommand{\bP}{\bm{P}}
\newcommand{\bQ}{\bm{Q}}
\newcommand{\bR}{\bm{R}}
\newcommand{\bS}{\bm{S}}
\newcommand{\bT}{\bm{T}}
\newcommand{\bU}{\bm{U}}
\newcommand{\bV}{\bm{V}}
\newcommand{\bW}{\bm{W}}
\newcommand{\bX}{\bm{X}}
\newcommand{\bY}{\bm{Y}}
\newcommand{\bZ}{\bm{Z}}

% denote vectors
\newcommand{\bbeta}{\bm{\beta}}
\newcommand{\balpha}{\bm{\alpha}}
\newcommand{\bgamma}{\bm{\gamma}}
\newcommand{\blambda}{\bm{\lambda}}
\newcommand{\bomega}{\bm{\omega}}
\newcommand{\bmu}{\bm{\mu}}
\newcommand{\bepsilon}{\bm{\epsilon}}
\newcommand{\btheta}{\bm{\theta}}
\newcommand{\bphi}{\bm{\phi}}
\newcommand{\bvarphi}{\bm{\varphi}}
\newcommand{\bxi}{\bm{\xi}}
\newcommand{\bpi}{\bm{\pi}}

% denote matrices
\newcommand{\bGamma}{\bm{\Gamma}}
\newcommand{\bLambda}{\bm{\Lambda}}
\newcommand{\bSigma}{\bm{\Sigma}}

% others
\newcommand{\bcE}{\bm{\mathcal{E}}}	% filtration
\newcommand{\bcF}{\bm{\mathcal{F}}}	% filtration
\newcommand{\bcG}{\bm{\mathcal{G}}}	% filtration


%----- double fonts -----%

\newcommand{\bbR}{\mathbb{R}}
\newcommand{\bbE}{\mathbb{E}}
\newcommand{\bbN}{\mathbb{N}}
\newcommand{\bbP}{\mathbb{P}}
\newcommand{\bbQ}{\mathbb{Q}}
\newcommand{\bbZ}{\mathbb{Z}}


%----- script fonts -----%

\newcommand{\cA}{\mathcal{A}}
\newcommand{\cB}{\mathcal{B}}
\newcommand{\cC}{\mathcal{C}}
\newcommand{\cD}{\mathcal{D}}
\newcommand{\cE}{\mathcal{E}}
\newcommand{\cF}{\mathcal{F}}
\newcommand{\cG}{\mathcal{G}}
\newcommand{\cH}{\mathcal{H}}
\newcommand{\cI}{\mathcal{I}}
\newcommand{\cJ}{\mathcal{J}}
\newcommand{\cK}{\mathcal{K}}
\newcommand{\cL}{\mathcal{L}}
\newcommand{\cM}{\mathcal{M}}
\newcommand{\cN}{\mathcal{N}}
\newcommand{\cO}{\mathcal{O}}
\newcommand{\cP}{\mathcal{P}}
\newcommand{\cQ}{\mathcal{Q}}
\newcommand{\cR}{\mathcal{R}}
\newcommand{\cS}{\mathcal{S}}
\newcommand{\cT}{\mathcal{T}}
\newcommand{\cU}{\mathcal{U}}
\newcommand{\cV}{\mathcal{V}}
\newcommand{\cW}{\mathcal{W}}
\newcommand{\cX}{\mathcal{X}}
\newcommand{\cY}{\mathcal{Y}}
\newcommand{\cZ}{\mathcal{Z}}


%----- special operators -----%

\newcommand{\argmin}{\mathop{\mathrm{argmin}}}
\newcommand{\argmax}{\mathop{\mathrm{argmax}}}

\newcommand{\bvar}{\textbf{Var}}
\newcommand{\bbias}{\textbf{Bias}}
\newcommand{\bcov}{\textbf{Cov}}
\newcommand{\bcor}{\textbf{Cor}}
\newcommand{\brank}{\textbf{rank}}
\newcommand{\bsign}{\textbf{sign}}
\newcommand{\bdiag}{\textbf{diag}}	% diagonal
\newcommand{\bdim}{\textbf{dim}}	% dimension
\newcommand{\btr}{\textbf{tr}}	    % trace
\newcommand{\bspan}{\textbf{span}}	% linear span
\newcommand{\bsupp}{\textbf{supp}}	% support
\newcommand{\bepi}{\textbf{epi}}	% epigraph

\newcommand{\perm}{\textbf{Perm}}	% permutation

\newcommand{\wass}{\textbf{Wass}}	% Wasserstein Distance
\newcommand{\ks}{\textbf{KS}}		% Kolomogov-Smirnov Distance

\newcommand{\brem}{\textbf{Rem}}		% remainders


\newcommand{\bzero}{{\mathbf{0}}}	% zero vector
\newcommand{\bone}{{\mathbf{1}}}	% all-one vector
\newcommand{\bbone}{{\mathbbm{1}}}	% indicator

\newcommand{\rmd}{\mathrm{d}}		% differentiation

\newcommand\indep{\protect\mathpalette{\protect\independenT}{\perp}}
\def\independenT#1#2{\mathrel{\rlap{$#1#2$}\mkern2mu{#1#2}}}	% independence

%----- distribution name -----%

\newcommand{\Exp}{\textbf{Exp}}
\newcommand{\Pois}{\textbf{Pois}}
\newcommand{\Gumb}{\textbf{Gumbel}}
\newcommand{\Bern}{\textbf{Bernoulli}}
\newcommand{\Bin}{\textbf{Bin}}
\newcommand{\NBin}{\textbf{NBin}}
\newcommand{\Multi}{\textbf{Multi}}
\newcommand{\Geo}{\textbf{Geo}}
\newcommand{\Hyper}{\textbf{Hyper}}
\newcommand{\SBM}{\textbf{SBM}}
\newcommand{\PoisProc}{\textbf{PoisProc}}

\usepackage{titling}

% Create subtitle command for use in maketitle
\newcommand{\subtitle}[1]{
	\posttitle{
		\begin{center}\large#1\end{center}
	}
}
 
\usepackage[margin=1in]{geometry} 
\usepackage{amsmath,amsthm,amssymb}
\usepackage{graphicx}
\usepackage{float}
\usepackage{enumerate,enumitem}

\newcommand{\N}{\mathbb{N}}
\newcommand{\Z}{\mathbb{Z}}
 
\newenvironment{theorem}[2][Theorem]{\begin{trivlist}
\item[\hskip \labelsep {\bfseries #1}\hskip \labelsep {\bfseries #2.}]}{\end{trivlist}}
\newenvironment{lemma}[2][Lemma]{\begin{trivlist}
\item[\hskip \labelsep {\bfseries #1}\hskip \labelsep {\bfseries #2.}]}{\end{trivlist}}
\newenvironment{exercise}[2][Exercise]{\begin{trivlist}
\item[\hskip \labelsep {\bfseries #1}\hskip \labelsep {\bfseries #2.}]}{\end{trivlist}}
\newenvironment{reflection}[2][Reflection]{\begin{trivlist}
\item[\hskip \labelsep {\bfseries #1}\hskip \labelsep {\bfseries #2.}]}{\end{trivlist}}
\newenvironment{proposition}[2][Proposition]{\begin{trivlist}
\item[\hskip \labelsep {\bfseries #1}\hskip \labelsep {\bfseries #2.}]}{\end{trivlist}}
\newenvironment{corollary}[2][Corollary]{\begin{trivlist}
\item[\hskip \labelsep {\bfseries #1}\hskip \labelsep {\bfseries #2.}]}{\end{trivlist}}

\usepackage[framed,numbered,autolinebreaks,useliterate]{mcode}

\def\name{Zhenghan Fang}

\usepackage{fancyhdr}
\pagestyle{fancy}
\fancyhf{}
\rhead{\name}
\cfoot{\thepage}
\renewcommand{\headrulewidth}{0pt}

\begin{document}

% --------------------------------------------------------------
%                         Start here
% --------------------------------------------------------------
 
%\renewcommand{\qedsymbol}{\filledbox}

\pagenumbering{arabic}

\title{STOR 614 - Linear Programming, Spring 2019 \\
Homework No. 6}
\author{\name}

\maketitle

\noindent
\textbf{Problem 1.}

\noindent
\textbf{(a)}

Convert to standard form
\begin{equation*}
  \begin{array}{cccccccc}
    \text{max}& z= &      -x_1    &   & -x_3 &  &   &      \\ 
    s.t       &    &    -2x_1      &   - x_2 &    +x_3 &    +s_1 &             &  =-5              \\
              &    &    -x_1      &    +2x_2 &    -2x_3 &     &     +s_2        &  = -2          \\
              &    &    x_1,      &    x_2, &    x_3, &    s_1, &     s_2        &  \ge 0          \\ 
  \end{array}
\end{equation*}

The initial tableau:
\begin{equation*}
  \begin{array}{cccccc|c|c}
    \hline
    z &  x_1      &  x_2 &  x_3 &  s_1 & s_2 &   \text{rhs} & \text{Basic var}  \\ \hline
    1 &    1      &    0 &    1 &    0 & 0    & 0        &  z=0              \\
    0 &    -2      &    -1 &    1 &    1 &    0 & -5        &  s_1 = -5          \\
    0 &    -1      &    2 &    -2 &    0 &    1 & -2        &  s_2 = -2          \\ \hline
  \end{array}
\end{equation*}

Iteration 1: 
\begin{equation*}
  \begin{array}{cccccc|c|c}
    \hline
    z &  x_1      &  x_2 &  x_3 &  s_1 & s_2 &   \text{rhs} & \text{Basic var}  \\ \hline
    1 &    1      &    0 &    1 &    0 & 0    & 0        &  z=0              \\
    0 &    -2      &    \boxed{-1} &    1 &    1 &    0 & -5        &  s_1 = -5          \\
    0 &    -1      &    2 &    -2 &    0 &    1 & -2        &  s_2 = -2          \\ \hline
    \text{ratio} &    1/2      &    0 &    NA &    NA &    NA &         &           \\ \hline
  \end{array}
\end{equation*}

Dual BFS: (0,0)

$s_1$ leaves and $x_2$ enters.


Iteration 2: 
\begin{equation*}
  \begin{array}{cccccc|c|c}
    \hline
    z &  x_1      &  x_2 &  x_3 &  s_1 & s_2 &   \text{rhs} & \text{Basic var}  \\ \hline
    1 &    1      &    0 &    1 &    0 & 0    & 0        &  z=0              \\
    0 &    2      &    1 &    -1 &    -1 &    0 & 5        &  x_2 = 5          \\
    0 &    \boxed{-5}      &    0 &    0 &    2 &    1 & -12        &  s_2 = -12          \\ \hline
    \text{ratio} &    1/5      &    NA &    NA &    NA &    NA &         &           \\ \hline
  \end{array}
\end{equation*}

Dual BFS: (0,0)

$s_2$ leaves and $x_1$ enters.

Iteration 3:
\begin{equation*}
  \begin{array}{cccccc|c|c}
    \hline
    z &  x_1      &  x_2 &  x_3 &  s_1 & s_2 &   \text{rhs} & \text{Basic var}  \\ \hline
    1 &    0      &    0 &    1 &    2/5 & 1/5    & -12/5        &  z=-12/5              \\
    0 &    0      &    1 &    -1 &    -1/5 &    2/5 & 1/5        &  x_2 = 1/5          \\
    0 &    1      &    0 &    0 &    -2/5 &    -1/5 & 12/5        &  x_1 = 12/5          \\ \hline
    \text{ratio} &    1/5      &    NA &    NA &    NA &    NA &         &           \\ \hline
  \end{array}
\end{equation*}

Dual BFS: (2/5,1/5)

No leaving variable to choose.

Terminate. $x=(12/5,1/5,0)$ is optimal, and the optimal value is -12/5.

\noindent
\textbf{(b)}

Convert to standard form.
\begin{equation*}
  \begin{array}{ccccccccccc}
    \text{max}& z= &    -3x_1      & -4x_2   & -2x_3 & -x_4   & -5x_5 & & &       \\ 
    s.t       &    &    x_1      &   -2x_2 &    -x_3 & +x_4                 &  +x_5 & +s_1 & & & =-3              \\
              &    &    -x_1      &   -x_2 &    -x_3 & +x_4                 &  +x_5 &  & +s_2 & & =-2              \\
              &    &    x_1      &    + x_2 & -2x_3    & +2x_4    & -3x_5 & & & +s_3 &           = 4          \\
              &    &    x_1,      &    x_2, & x_3, & x_4, & x_5, &   s_1, &    s_2  & s_3 &             \ge 0\\ 
  \end{array}
\end{equation*}

The initial tableau:
\begin{equation*}
  \begin{array}{ccccccccc|c|c}
    \hline
    z &  x_1      &  x_2 &  x_3 & x_4 & x_5 & s_1 & s_2 & s_3 &  \text{rhs} & \text{Basic var}  \\ \hline
    1 &    3      &    4 &    2 &    1 & 5    & 0 & 0 & 0 &     0  &  z=0              \\
    0 &    1      &    -2 &    -1 &    1 &    1 & 1 & 0 & 0 & -3      &  s_1 = -3          \\
    0 &    -1      &    -1 &    -1 &    1 &    1 & 0 & 1 & 0 & -2   &  s_2 = -2          \\ 
    0 &    1      &    1 &    -2 &    2 &    -3 & 0 & 0 & 1 & 4   &  s_3 = 4          \\ \hline
  \end{array}
\end{equation*}

Iteration 1: 
\begin{equation*}
  \begin{array}{ccccccccc|c|c}
    \hline
    z &  x_1      &  x_2 &  x_3 & x_4 & x_5 & s_1 & s_2 & s_3 &  \text{rhs} & \text{Basic var}  \\ \hline
    1 &    3      &    4 &    2 &    1 & 5    & 0 & 0 & 0 &     0  &  z=0              \\
    0 &    1      &    -2 &    \boxed{-1} &    1 &    1 & 1 & 0 & 0 & -3      &  s_1 = -3          \\
    0 &    -1      &    -1 &    -1 &    1 &    1 & 0 & 1 & 0 & -2   &  s_2 = -2          \\ 
    0 &    1      &    1 &    -2 &    2 &    -3 & 0 & 0 & 1 & 4   &  s_3 = 4          \\ \hline
    \text{ratio} &    NA      &    2 &    2 &    NA &    NA & NA & NA & NA                    \\ \hline
  \end{array}
\end{equation*}

Dual BFS: (0,0,0)

$s_1$ leaves and $x_3$ enters.

Iteration 2: 
\begin{equation*}
  \begin{array}{ccccccccc|c|c}
    \hline
    z &  x_1      &  x_2 &  x_3 & x_4 & x_5 & s_1 & s_2 & s_3 &  \text{rhs} & \text{Basic var}  \\ \hline
    1 &    5      &    0 &    0 &    3 & 7    & 2 & 0 & 0 &     -6  &  z=-6              \\
    0 &    -1      &    2 &    1 &    -1 &    -1 & -1 & 0 & 0 & 3      &  x_3 = 3          \\
    0 &    -2      &    1 &    0  &    0 &    0 & -1 & 1 & 0 & 1   &  s_2 = 1          \\ 
    0 &    -1      &    5 &    0 &    0 &    -5 & -2 & 0 & 1 & 10   &  s_3 = 10          \\ \hline
    \text{ratio} &    NA      &    2 &    2 &    NA &    NA & NA & NA & NA                    \\ \hline
  \end{array}
\end{equation*}

Dual BFS: (2,0,0)

No leaving variable to choose.

Terminate. $x=(0,0,3,0,0)$ is optimal. The optimal value is -6.


\vspace{\baselineskip}
\noindent
\textbf{Problem 2.}
\begin{proof}
If (a) holds, then (b) cannot hold, because
\[ p^T A >0, x\ge0, x\ne0 \implies p^T Ax > 0, \] which contradicts with $Ax=0$.

Suppose that (a) does not hold. Let $ \bm{1} = \begin{bmatrix} 1 \\ 1 \\ \vdots \\ 1 \end{bmatrix}$. Consider the pair
\begin{equation*}
  \begin{array}{llllllll}
    \text{Primal:} &\text{max}& \bm{1}^Tx &          &   & \text{Dual:} & \text{min} & 0       \\ 
    & s.t       &  Ax=0  &          & &&  s.t  &   p^T A \ge \bm{1}           \\
    &           & x\ge 0   &         
  \end{array}
\end{equation*}
The primal has a unique optimal solution $x=0$ and the optimal value is 0. Then the dual is feasible, so there exists $p$ such that $p^TA \ge \bm{1} > 0$. 
\end{proof}

\vspace{\baselineskip}
\noindent
\textbf{Problem 3.}
\begin{proof}
If (b) holds, then \[a^Tx \le \sum_{i=1}^m \lambda_i a_i^Tx \le \sum_{i=1}^m \lambda_i (\max_{i=1,...,m} a_i^Tx) = \max_{i=1,...,m} {a_i^Tx} \] for any $x\ge0$.

Suppose that (a) holds. Consider the pair
\begin{equation*}
  \begin{array}{llllllll}
    \text{Primal:} &\text{max}& a^Tx &          &   & \text{Dual:} & \text{min} & \sum_{i=1}^m p_i       \\ 
    & s.t       &  a_i^Tx\le1,\quad i=1,...,m  &          & &&  s.t  &   \sum_{i=1}^m p_ia_i \ge a           \\
    &           & x\ge 0   &  & & & & p_i \ge 0, \quad i=1,...,m      
  \end{array}
\end{equation*}
The primal is not infeasible ($x=0$ is a trivial feasible solution). Thus, the primal has an optimal solution and the optimal value is bounded by $1$. Thus, the optimal value of the dual is $\le 1$. Thus, there exist coefficients $p_i$'s, such that
\begin{gather*}
    \sum_{i=1}^m p_i \le 1, \\
    \sum_{i=1}^m p_ia_i \ge a, \\ \tag{3.1}
    p_i \ge 0, \quad i=1,...,m.
\end{gather*}

Furthermore, consider the pair
\begin{equation*}
  \begin{array}{llllllll}
    \text{Primal:} &\text{max}& a^Tx &          &   & \text{Dual:} & \text{min} & \sum_{i=1}^m -q_i       \\ 
    & s.t       &  a_i^Tx\le -1,\quad i=1,...,m  &          & &&  s.t  &   \sum_{i=1}^m q_ia_i \ge a           \\
    &           & x\ge 0   &  & & & & q_i \ge 0, \quad i=1,...,m      
  \end{array} 
\end{equation*}
If the primal is infeasible, then the dual is unbounded (dual cannot be infeasible because of (3.1)). If the primal is feasible, then the primal optimal value is $\le -1$. Therefore, either way, there exist coefficients $q_i$'s, such that 
\begin{gather*}
    \sum_{i=1}^m -q_i \le -1 \implies \sum_{i=1}^m q_i \ge 1, \\
    \sum_{i=1}^m q_ia_i \ge a, \\ \tag{3.2}
    q_i \ge 0, \quad i=1,...,m.
\end{gather*}

From (3.1) and (3.2), and that the feasible set of dual problem is convex, we get statement (b).
\end{proof}

\vspace{\baselineskip}
\noindent
\textbf{Problem 4.}

(a) The optimal solution is $(0,25,25,0,0)$. The optimal objective function value is 300.

(b) The dual LP:
\begin{equation*}
  \begin{array}{cccccccc}
    \text{min}&    &      50y_1    & +100y_2  &       \\ 
    s.t       &    &    y_1      &   +2y_2 &    \ge 3      \\
              &    &    y_1      &    +3y_2 &   \ge 7        \\
              &    &    y_1      &    +y_2 &   \ge 5   \\ 
              &    &    y_1      &        &   \ge 0  \\ 
              &    &          &   y_2     &   \ge 0  \\ 
  \end{array}
\end{equation*}

A dual optimal solution is $(y_1,y_2) = (4,1)$. The dual has only one optimal solution, because the primal LP has a nondegenerate optimal BFS.

(c) The reduced cost for $x_1$ will be $3-\Delta$. For $\Delta \le 3$, $\{x_3,x_2\}$ continues to be an optimal basis. For $\Delta \le 3$, an optimal solution is $(0,25,25)$, and the optimal value is 300. For $\Delta = 4$, the tableau is
\begin{equation*}
  \begin{array}{cccccc|c|c|c}
    \hline
    z &  x_1      &  x_2 &  x_3 &  s_1 & s_2 &   \text{rhs} & \text{Basic var} & \text{ratio}  \\ \hline
    1 &    -1     &    0 &    0 &    4 & 1    & 300        &  z=300              \\
    0 &    0.5      &    0 &    1 &    1.5 &    -0.5 & 25        &  x_3 = 25   & 50       \\
    0 &    0.5      &    1 &    0 &    -0.5 &    0.5 & 25        &  x_2 = 25    & 50      \\ \hline
  \end{array}
\end{equation*}

$x_1$ enters and $x_3$ leaves.
\begin{equation*}
  \begin{array}{cccccc|c|c}
    \hline
    z &  x_1      &  x_2 &  x_3 &  s_1 & s_2 &   \text{rhs} & \text{Basic var}  \\ \hline
    1 &    0     &    0 &    2 &    7 & 0    & 350        &  z=350              \\
    0 &    1      &    0 &    2 &    3 &    -1 & 50        &  x_1 = 50    \\
    0 &    0      &    1 &    -1 &    -2 &    1 & 0        &  x_2 = 0      \\ \hline
  \end{array}
\end{equation*}

An optimal solution is $(50,0,0,0,0)$, and the optimal value is 350.

(d)
\begin{equation*}
  \begin{array}{cccccc|c|c}
    \hline
    z &  x_1      &  x_2 &  x_3 &  s_1 & s_2 &   \text{rhs} & \text{Basic var}   \\ \hline
    1 &    3     &    0 &    -\Delta &    4-\Theta & 1    & 300        &  z=300              \\
    0 &    0.5      &    0 &    1 &    1.5 &    -0.5 & 25        &  x_3 = 25         \\
    0 &    0.5      &    1 &    0 &    -0.5 &    0.5 & 25        &  x_2 = 25          \\ \hline
  \end{array}
\end{equation*}

\begin{equation*}
  \begin{array}{cccccc|c|c}
    \hline
    z &  x_1      &  x_2 &  x_3 &  s_1 & s_2 &   \text{rhs} & \text{Basic var}   \\ \hline
    1 &    3 +0.5\Delta     &    0 &    0 &    4-\Theta +1.5\Delta & 1-0.5\Delta    & 300 +25\Delta       &  z=300+25\Delta           \\
    0 &    0.5      &    0 &    1 &    1.5 &    -0.5 & 25        &  x_3 = 25         \\
    0 &    0.5      &    1 &    0 &    -0.5 &    0.5 & 25        &  x_2 = 25          \\ \hline
  \end{array}
\end{equation*}
For the current basis to remain optimal, we need
\begin{align*}
    & 3 +0.5\Delta \ge 0\\
    & 4-\Theta +1.5\Delta \ge 0 \\
    & 1-0.5\Delta \ge 0
\end{align*}
Thus, for $-6 \le \Delta \le 2$ and $\Theta \le 4+1.5\Delta$, this tableau shows an optimal solution. Optimal solution: $(0,25,25,0,0)$, optimal value: $300+25\Delta$. $(\Delta,\Theta) = (2,2)$ belongs to that range.

(e)
\begin{equation*}
  \begin{array}{cccccc|c|c}
    \hline
    z &  x_1      &  x_2 &  x_3 &  s_1 & s_2 &   \text{rhs} & \text{Basic var}   \\ \hline
    1 &    3     &    0 &    0 &    4 & 1    & 300+4\Delta       &  z=300+4\Delta              \\
    0 &    0.5      &    0 &    1 &    1.5 &    -0.5 & 25+1.5\Delta        &  x_3 = 25+1.5\Delta         \\
    0 &    0.5      &    1 &    0 &    -0.5 &    0.5 & 25-0.5\Delta        &  x_2 = 25-0.5\Delta           \\ \hline
  \end{array}
\end{equation*}
For the current basis to remain optimal, we need
\begin{align*}
    & 25+1.5\Delta \ge 0 \\
    & 25-0.5\Delta \ge 0
\end{align*}
Thus, for $-50/3 \le \Delta \le 50$, the current basis continue to be optimal. The optimal value is $300+4\Delta$. When $\Delta = 100$,
\begin{equation*}
  \begin{array}{cccccc|c|c}
    \hline
    z &  x_1      &  x_2 &  x_3 &  s_1 & s_2 &   \text{rhs} & \text{Basic var}   \\ \hline
    1 &    3     &    0 &    0 &    4 & 1    & 700       &  z=700              \\
    0 &    0.5      &    0 &    1 &    1.5 &    -0.5 & 175        &  x_3 = 175         \\
    0 &    0.5      &    1 &    0 &    -0.5 &    0.5 & -25        &  x_2 = -25           \\ \hline
  \end{array}
\end{equation*}
$x_2$ leaves and $s_1$ enters.
\begin{equation*}
  \begin{array}{cccccc|c|c}
    \hline
    z &  x_1      &  x_2 &  x_3 &  s_1 & s_2 &   \text{rhs} & \text{Basic var}   \\ \hline
    1 &    7     &    8 &    0 &    0 & 5    & 500       &  z=500              \\
    0 &    2      &    3 &    1 &    0 &    1 & 100        &  x_3 = 100         \\
    0 &    -1      &    -2 &    0 &    1 &    -1 & 50        &  s_1 = 50           \\ \hline
  \end{array}
\end{equation*}
An optimal solution is $(0,0,100,50,0)$, and the optimal value is $500$.

\vspace{\baselineskip}
\noindent
\textbf{Problem 5.}

(a) Basis: $\{x_1,x_6,x_3\}$. $A_B =\begin{bmatrix}6 & 0 & 5 \\ 3 & 1 & 3 \\ 3 & 0 & 5\end{bmatrix} $. $A_B^{-1} = \begin{bmatrix}1/3 & 0 & -1/3 \\ -2/5 & 1 & -1/5 \\ -1/5 & 0 & 2/5\end{bmatrix}$. $c_B=\begin{bmatrix}3 \\ 0 \\ 4\end{bmatrix}$. BFS: $(5/3,0,3,0,0,1,0)$. Objective function value = 17.

(b) The largest value is 3. The optimal value is 17. If $x_4,x_5,x_7$ are zero, the optimal value is reached. Thus the set of optimal solutions is $\{\begin{bmatrix}1/3\lambda+5/3 \\ \lambda \\3-\lambda \\0 \\ 0 \\1 \\0 \end{bmatrix} \mid 0 \le \lambda \le 3\}$.

(c)
If the coefficient of $x_3$ is $4+\Delta$, the tableau will be
\begin{equation*}
  \begin{array}{cccccccc|c|c}
    \hline
    z &  x_1      &  x_2 &  x_3 &  x_4 & x_5 & x_6 & x_7 &   \text{rhs} & \text{Basic var}   \\ \hline
    1 &    0     &    2 &    -\Delta &    1 & 1/5    & 0 & 3/5 & 17      &  z=17              \\
    0 &    1      &    -1/3 &    0 &    2/3 &    1/3 & 0 & -1/3 & 5/3        &  x_3 = 175         \\
    0 &    0      &    0 &    0 &    -1 &    -2/5 & 1 & -1/5 & 1       &  x_2 = -25           \\
    0 &    0      &    1 &    1 &    0 &    -1/5 & 0 & 2/5 & 3       &  x_2 = -25           \\ \hline
  \end{array}
\end{equation*}

\begin{equation*}
  \begin{array}{cccccccc|c|c}
    \hline
    z &  x_1      &  x_2 &  x_3 &  x_4 & x_5 & x_6 & x_7 &   \text{rhs} & \text{Basic var}   \\ \hline
    1 &    0     &    2+\Delta &    0 &    1 & 1/5 - 1/5\Delta    & 0 & 3/5+2/5\Delta & 17+\Delta      &  z=17+\Delta              \\
    0 &    1      &    -1/3 &    0 &    2/3 &    1/3 & 0 & -1/3 & 5/3        &  x_3 = 175         \\
    0 &    0      &    0 &    0 &    -1 &    -2/5 & 1 & -1/5 & 1       &  x_2 = -25           \\
    0 &    0      &    1 &    1 &    0 &    -1/5 & 0 & 2/5 & 3       &  x_2 = -25           \\ \hline
  \end{array}
\end{equation*}
Thus, the smallest value is $4-1.5=2.5$. The optimal value is $15.5$. The optimal value is reached if $x_2,x_4,x_5=0$. Thus, the set of optimal solutions is $\{\begin{bmatrix}  1/3\lambda+5/3 \\ 0 \\ -2/5\lambda+3 \\ 0 \\ 0 \\ 1/5\lambda+1 \\ \lambda \end{bmatrix}\mid 0\le\lambda\le 5/2\}$.

(d) If the value is $25+\Delta$, the tableau will be
\begin{equation*}
  \begin{array}{cccccccc|c|c}
    \hline
    z &  x_1      &  x_2 &  x_3 &  x_4 & x_5 & x_6 & x_7 &   \text{rhs} & \text{Basic var}   \\ \hline
    1 &    0     &    2 &   0 &    1 & 1/5    & 0 & 3/5 & 17+1/5\Delta      &  z=17              \\
    0 &    1      &    -1/3 &    0 &    2/3 &    1/3 & 0 & -1/3 & 5/3+1/3\Delta        &  x_3 = 175         \\
    0 &    0      &    0 &    0 &    -1 &    -2/5 & 1 & -1/5 & 1-2/5\Delta       &  x_2 = -25           \\
    0 &    0      &    1 &    1 &    0 &    -1/5 & 0 & 2/5 & 3-1/5\Delta       &  x_2 = -25           \\ \hline
  \end{array}
\end{equation*}
Thus, the largest value is $27.5$. The new optimal basis will be $\{x_1,x_5,x_3\}$.

(e) Because the dual optimal solution is $(1/5,0,3/5)$, the rate of change of the primal optimal value in response to a change in the largest value of resource 1, 2, 3 are $1/5,0,3/5$, respectively. Thus, if we purchase $n$ packages of resources, the primal optimal value increases by $3n\times1/5 + 4n\times3/5 = 3n$, and the cost increases by $5n$. The increase of cost is greater than the increase of earnings, so we should not purchase any of this package.


% --------------------------------------------------------------
%     You don't have to mess with anything below this line.
% --------------------------------------------------------------

\end{document}