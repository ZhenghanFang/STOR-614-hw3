% --------------------------------------------------------------
% This is all preamble stuff that you don't have to worry about.
% Head down to where it says "Start here"
% --------------------------------------------------------------
 
\documentclass[12pt]{article}
\usepackage{amsgen,amsmath,amstext,amsbsy,amsopn,amssymb,mathabx,amsthm,bm,bbm,romannum,enumitem}
\usepackage[dvips]{graphicx}

\usepackage[pagebackref,bookmarksnumbered]{hyperref}
\usepackage{url}
\hypersetup{
	colorlinks=true,
	linkcolor=red,
	filecolor=magenta,      
	urlcolor=blue,
}

\setcounter{tocdepth}{3}
\usepackage[depth=3]{bookmark}

\usepackage[margin=1in]{geometry}
\renewcommand{\baselinestretch}{1.5}	% Line Stretch

\usepackage[utf8]{inputenc}

%----- theorems -----%

\newtheorem{thm}{Theorem}[section]
\newtheorem{lem}[thm]{Lemma}
\newtheorem{prop}[thm]{Proposition}
\newtheorem{coro}[thm]{Corollary}

\theoremstyle{definition}
\newtheorem{dfn}{Definition}[section]
\newtheorem*{pchln}{Punchline}

\theoremstyle{remark}
\newtheorem*{rmk}{Remark}
\newtheorem*{note}{Note}
\newtheorem{eg}{Example}[section]
\newtheorem{fact}{Fact}[section]
\newtheorem*{hint}{Hint}


%----- bold fonts -----%

\newcommand{\ab}{\mathbf{a}}
\newcommand{\bbb}{\mathbf{b}}
\newcommand{\cbb}{\mathbf{c}}
\newcommand{\db}{\mathbf{d}}
\newcommand{\eb}{\mathbf{e}}
\newcommand{\fb}{\mathbf{f}}
\newcommand{\gb}{\mathbf{g}}
\newcommand{\hb}{\mathbf{h}}
\newcommand{\ib}{\mathbf{i}}
\newcommand{\jb}{\mathbf{j}}
\newcommand{\kb}{\mathbf{k}}
\newcommand{\lb}{\mathbf{l}}
\newcommand{\mb}{\mathbf{m}}
\newcommand{\nbb}{\mathbf{n}}
\newcommand{\ob}{\mathbf{o}}
\newcommand{\pb}{\mathbf{p}}
\newcommand{\qb}{\mathbf{q}}
\newcommand{\rb}{\mathbf{r}}
\newcommand{\sbb}{\mathbf{s}}
\newcommand{\tb}{\mathbf{t}}
\newcommand{\ub}{\mathbf{u}}
\newcommand{\vb}{\mathbf{v}}
\newcommand{\wb}{\mathbf{w}}
\newcommand{\xb}{\mathbf{x}}
\newcommand{\yb}{\mathbf{y}}
\newcommand{\zb}{\mathbf{z}}

% denote vectors
\newcommand{\ba}{\bm{a}}
\newcommand{\bb}{\bm{b}}
\newcommand{\bc}{\bm{c}}
\newcommand{\bd}{\bm{d}}
\newcommand{\be}{\bm{e}}
\newcommand{\bbf}{\bm{f}}
\newcommand{\bg}{\bm{g}}
\newcommand{\bh}{\bm{h}}
\newcommand{\bi}{\bmf{i}}
\newcommand{\bj}{\bm{j}}
\newcommand{\bk}{\bm{k}}
\newcommand{\bl}{\bm{l}}
\newcommand{\bbm}{\bm{m}}
\newcommand{\bn}{\bm{n}}
\newcommand{\bo}{\bm{o}}
\newcommand{\bp}{\bm{p}}
\newcommand{\bq}{\bm{q}}
\newcommand{\br}{\bm{r}}
\newcommand{\bs}{\bm{s}}
\newcommand{\bt}{\bm{t}}
\newcommand{\bu}{\bm{u}}
\newcommand{\bv}{\bm{v}}
\newcommand{\bw}{\bm{w}}
\newcommand{\bx}{\bm{x}}
\newcommand{\by}{\bm{y}}
\newcommand{\bz}{\bm{z}}

% denote random matrices
\newcommand{\Ab}{\mathbf{A}}
\newcommand{\Bb}{\mathbf{B}}
\newcommand{\Cb}{\mathbf{C}}
\newcommand{\Db}{\mathbf{D}}
\newcommand{\Eb}{\mathbf{E}}
\newcommand{\Fb}{\mathbf{F}}
\newcommand{\Gb}{\mathbf{G}}
\newcommand{\Hb}{\mathbf{H}}
\newcommand{\Ib}{\mathbf{I}}
\newcommand{\Jb}{\mathbf{J}}
\newcommand{\Kb}{\mathbf{K}}
\newcommand{\Lb}{\mathbf{L}}
\newcommand{\Mb}{\mathbf{M}}
\newcommand{\Nb}{\mathbf{N}}
\newcommand{\Ob}{\mathbf{O}}
\newcommand{\Pb}{\mathbf{P}}
\newcommand{\Qb}{\mathbf{Q}}
\newcommand{\Rb}{\mathbf{R}}
\newcommand{\Sbb}{\mathbf{S}}
\newcommand{\Tb}{\mathbf{T}}
\newcommand{\Ub}{\mathbf{U}}
\newcommand{\Vb}{\mathbf{V}}
\newcommand{\Wb}{\mathbf{W}}
\newcommand{\Xb}{\mathbf{X}}
\newcommand{\Yb}{\mathbf{Y}}
\newcommand{\Zb}{\mathbf{Z}}

% denote random vectors
\newcommand{\bA}{\bm{A}}
\newcommand{\bB}{\bm{B}}
\newcommand{\bC}{\bm{C}}
\newcommand{\bD}{\bm{D}}
\newcommand{\bE}{\bm{E}}
\newcommand{\bF}{\bm{F}}
\newcommand{\bG}{\bm{G}}
\newcommand{\bH}{\bm{H}}
\newcommand{\bI}{\bm{I}}
\newcommand{\bJ}{\bm{J}}
\newcommand{\bK}{\bm{K}}
\newcommand{\bL}{\bm{L}}
\newcommand{\bM}{\bm{M}}
\newcommand{\bN}{\bm{N}}
\newcommand{\bO}{\bm{O}}
\newcommand{\bP}{\bm{P}}
\newcommand{\bQ}{\bm{Q}}
\newcommand{\bR}{\bm{R}}
\newcommand{\bS}{\bm{S}}
\newcommand{\bT}{\bm{T}}
\newcommand{\bU}{\bm{U}}
\newcommand{\bV}{\bm{V}}
\newcommand{\bW}{\bm{W}}
\newcommand{\bX}{\bm{X}}
\newcommand{\bY}{\bm{Y}}
\newcommand{\bZ}{\bm{Z}}

% denote vectors
\newcommand{\bbeta}{\bm{\beta}}
\newcommand{\balpha}{\bm{\alpha}}
\newcommand{\bgamma}{\bm{\gamma}}
\newcommand{\blambda}{\bm{\lambda}}
\newcommand{\bomega}{\bm{\omega}}
\newcommand{\bmu}{\bm{\mu}}
\newcommand{\bepsilon}{\bm{\epsilon}}
\newcommand{\btheta}{\bm{\theta}}
\newcommand{\bphi}{\bm{\phi}}
\newcommand{\bvarphi}{\bm{\varphi}}
\newcommand{\bxi}{\bm{\xi}}
\newcommand{\bpi}{\bm{\pi}}

% denote matrices
\newcommand{\bGamma}{\bm{\Gamma}}
\newcommand{\bLambda}{\bm{\Lambda}}
\newcommand{\bSigma}{\bm{\Sigma}}

% others
\newcommand{\bcE}{\bm{\mathcal{E}}}	% filtration
\newcommand{\bcF}{\bm{\mathcal{F}}}	% filtration
\newcommand{\bcG}{\bm{\mathcal{G}}}	% filtration


%----- double fonts -----%

\newcommand{\bbR}{\mathbb{R}}
\newcommand{\bbE}{\mathbb{E}}
\newcommand{\bbN}{\mathbb{N}}
\newcommand{\bbP}{\mathbb{P}}
\newcommand{\bbQ}{\mathbb{Q}}
\newcommand{\bbZ}{\mathbb{Z}}


%----- script fonts -----%

\newcommand{\cA}{\mathcal{A}}
\newcommand{\cB}{\mathcal{B}}
\newcommand{\cC}{\mathcal{C}}
\newcommand{\cD}{\mathcal{D}}
\newcommand{\cE}{\mathcal{E}}
\newcommand{\cF}{\mathcal{F}}
\newcommand{\cG}{\mathcal{G}}
\newcommand{\cH}{\mathcal{H}}
\newcommand{\cI}{\mathcal{I}}
\newcommand{\cJ}{\mathcal{J}}
\newcommand{\cK}{\mathcal{K}}
\newcommand{\cL}{\mathcal{L}}
\newcommand{\cM}{\mathcal{M}}
\newcommand{\cN}{\mathcal{N}}
\newcommand{\cO}{\mathcal{O}}
\newcommand{\cP}{\mathcal{P}}
\newcommand{\cQ}{\mathcal{Q}}
\newcommand{\cR}{\mathcal{R}}
\newcommand{\cS}{\mathcal{S}}
\newcommand{\cT}{\mathcal{T}}
\newcommand{\cU}{\mathcal{U}}
\newcommand{\cV}{\mathcal{V}}
\newcommand{\cW}{\mathcal{W}}
\newcommand{\cX}{\mathcal{X}}
\newcommand{\cY}{\mathcal{Y}}
\newcommand{\cZ}{\mathcal{Z}}


%----- special operators -----%

\newcommand{\argmin}{\mathop{\mathrm{argmin}}}
\newcommand{\argmax}{\mathop{\mathrm{argmax}}}

\newcommand{\bvar}{\textbf{Var}}
\newcommand{\bbias}{\textbf{Bias}}
\newcommand{\bcov}{\textbf{Cov}}
\newcommand{\bcor}{\textbf{Cor}}
\newcommand{\brank}{\textbf{rank}}
\newcommand{\bsign}{\textbf{sign}}
\newcommand{\bdiag}{\textbf{diag}}	% diagonal
\newcommand{\bdim}{\textbf{dim}}	% dimension
\newcommand{\btr}{\textbf{tr}}	    % trace
\newcommand{\bspan}{\textbf{span}}	% linear span
\newcommand{\bsupp}{\textbf{supp}}	% support
\newcommand{\bepi}{\textbf{epi}}	% epigraph

\newcommand{\perm}{\textbf{Perm}}	% permutation

\newcommand{\wass}{\textbf{Wass}}	% Wasserstein Distance
\newcommand{\ks}{\textbf{KS}}		% Kolomogov-Smirnov Distance

\newcommand{\brem}{\textbf{Rem}}		% remainders


\newcommand{\bzero}{{\mathbf{0}}}	% zero vector
\newcommand{\bone}{{\mathbf{1}}}	% all-one vector
\newcommand{\bbone}{{\mathbbm{1}}}	% indicator

\newcommand{\rmd}{\mathrm{d}}		% differentiation

\newcommand\indep{\protect\mathpalette{\protect\independenT}{\perp}}
\def\independenT#1#2{\mathrel{\rlap{$#1#2$}\mkern2mu{#1#2}}}	% independence

%----- distribution name -----%

\newcommand{\Exp}{\textbf{Exp}}
\newcommand{\Pois}{\textbf{Pois}}
\newcommand{\Gumb}{\textbf{Gumbel}}
\newcommand{\Bern}{\textbf{Bernoulli}}
\newcommand{\Bin}{\textbf{Bin}}
\newcommand{\NBin}{\textbf{NBin}}
\newcommand{\Multi}{\textbf{Multi}}
\newcommand{\Geo}{\textbf{Geo}}
\newcommand{\Hyper}{\textbf{Hyper}}
\newcommand{\SBM}{\textbf{SBM}}
\newcommand{\PoisProc}{\textbf{PoisProc}}

\usepackage{titling}

% Create subtitle command for use in maketitle
\newcommand{\subtitle}[1]{
	\posttitle{
		\begin{center}\large#1\end{center}
	}
}
 
\usepackage[margin=1in]{geometry} 
\usepackage{amsmath,amsthm,amssymb}
\usepackage{graphicx}
\usepackage{float}

\newcommand{\N}{\mathbb{N}}
\newcommand{\Z}{\mathbb{Z}}
 
\newenvironment{theorem}[2][Theorem]{\begin{trivlist}
\item[\hskip \labelsep {\bfseries #1}\hskip \labelsep {\bfseries #2.}]}{\end{trivlist}}
\newenvironment{lemma}[2][Lemma]{\begin{trivlist}
\item[\hskip \labelsep {\bfseries #1}\hskip \labelsep {\bfseries #2.}]}{\end{trivlist}}
\newenvironment{exercise}[2][Exercise]{\begin{trivlist}
\item[\hskip \labelsep {\bfseries #1}\hskip \labelsep {\bfseries #2.}]}{\end{trivlist}}
\newenvironment{reflection}[2][Reflection]{\begin{trivlist}
\item[\hskip \labelsep {\bfseries #1}\hskip \labelsep {\bfseries #2.}]}{\end{trivlist}}
\newenvironment{proposition}[2][Proposition]{\begin{trivlist}
\item[\hskip \labelsep {\bfseries #1}\hskip \labelsep {\bfseries #2.}]}{\end{trivlist}}
\newenvironment{corollary}[2][Corollary]{\begin{trivlist}
\item[\hskip \labelsep {\bfseries #1}\hskip \labelsep {\bfseries #2.}]}{\end{trivlist}}

\usepackage[framed,numbered,autolinebreaks,useliterate]{mcode}

\def\name{Zhenghan Fang}

\usepackage{fancyhdr}
\pagestyle{fancy}
\fancyhf{}
\rhead{\name}
\cfoot{\thepage}
\renewcommand{\headrulewidth}{0pt}

\begin{document}

% --------------------------------------------------------------
%                         Start here
% --------------------------------------------------------------
 
%\renewcommand{\qedsymbol}{\filledbox}

\pagenumbering{arabic}

\title{STOR 614 - Linear Programming, Spring 2019 \\
Homework No. 4}
\author{\name}

\maketitle

\noindent
\textbf{Problem 1.}

\noindent
\textbf{(a)}

Phase I.

Add artificial variables $y_1,y_2$.
\begin{equation*}
  \begin{array}{cccccccc}
    \text{max}& t= &          &   & &  -y_1 &  -y_2 &      \\ 
    s.t       &    &    2x_1      &   + x_2 &    +x_3 &    +y_1 &             &  =4              \\
              &    &    x_1      &    +x_2 &    +2x_3 &     &     +y_2        &  = 2          \\
              &    &    x_1,      &    x_2, &    x_3, &    y_1, &     y_2        &  \ge 0          \\ 
  \end{array}
\end{equation*}

Transform LP to canonical form.
\begin{equation*}
  \begin{array}{cccccccc}
    \text{max}& t= &    3x_1      &+2x_2   & +3x_3 &  -6 &   &      \\ 
    s.t       &    &    2x_1      &   + x_2 &    +x_3 &    +y_1 &             &  =4              \\
              &    &    x_1      &    +x_2 &    +2x_3 &     &     +y_2        &  = 2          \\
              &    &    x_1,      &    x_2, &    x_3, &    y_1, &     y_2        &  \ge 0          \\ 
  \end{array}
\end{equation*}

The initial tableau is as follows
\begin{equation*}
  \begin{array}{cccccc|c|c}
    \hline
    t &  x_1      &  x_2 &  x_3 &  y_1 & y_2 &   \text{rhs} & \text{Basic var}  \\ \hline
    1 &    -3      &    -2 &    -3 &    0 & 0    & -6        &  t=-6              \\
    0 &    2      &    1 &    1 &    1 &    0 & 4        &  y_1 = 4          \\
    0 &    1      &    1 &    2 &    0 &    1 & 2        &  y_2 = 2          \\ \hline
  \end{array}
\end{equation*}

First iteration: $x_1$ enters and $y_1$ leaves.
\begin{equation*}
  \begin{array}{cccccc|c|c}
    \hline
    t &  x_1      &  x_2 &  x_3 &  y_1 & y_2 &   \text{rhs} & \text{Basic var}  \\ \hline
    1 &    0      &    -1/2 &    -3/2 &    3/2 & 0    & 0        &  t=0              \\
    0 &    1      &    1/2 &    1/2 &    1/2 &    0 & 2        &  x_1 = 2          \\
    0 &    0      &    1/2 &    3/2 &   -1/2 &    1 & 0        &  y_2 = 0          \\ \hline
  \end{array}
\end{equation*}

Second iteration: $x_2$ enters and $y_2$ leaves.
\begin{equation*}
  \begin{array}{cccccc|c|c}
    \hline
    t &  x_1      &  x_2 &  x_3 &  y_1 & y_2 &   \text{rhs} & \text{Basic var}  \\ \hline
    1 &    0      &    0   &    0     &    1   & 1    & 0        &  t=0              \\
    0 &    1      &    0   &    -1    &    1   &   -1 & 2        &  x_1 = 2          \\
    0 &    0      &    1   &    3     &   -1   &    2 & 0        &  x_2 = 0          \\ \hline
  \end{array}
\end{equation*}

The optimal $t=0$. Case 2.1. Obtain a simplex tableau for the original LP.
\begin{equation*}
  \begin{array}{cccc|c|c}
    \hline
    z &  x_1      &  x_2   &  x_3       &   \text{rhs} & \text{Basic var}  \\ \hline
    1 &    0      &    0   &    2       & 2            &  z=2              \\
    0 &    1      &    0   &    -1      & 2            &  x_1 = 2          \\
    0 &    0      &    1   &    3       & 0            &  x_2 = 0          \\ \hline
  \end{array}
\end{equation*}

First iteration: $ x_3 $ enters and $ x_2 $ leaves.
\begin{equation*}
  \begin{array}{cccc|c|c}
    \hline
    z &  x_1      &  x_2   &  x_3       &   \text{rhs} & \text{Basic var}  \\ \hline
    1 &    0      &  -2/3  &    0       & 2            &  z=2              \\
    0 &    1      &  1/3   &    0       & 2            &  x_1 = 2          \\
    0 &    0      &  1/3   &    1       & 0            &  x_3 = 0          \\ \hline
  \end{array}
\end{equation*}

The original LP has an optimal solution $(x_1,x_2,x_3) = (2,0,0)$ and the optimal value is $z=2$.

\noindent
\textbf{(b)}

Convert to standard form.
\begin{equation*}
  \begin{array}{ccccccc}
    \text{min}& z= &    2x_1      &+3x_2   & &   &        \\ 
    s.t       &    &    1/2x_1      &   + 1/4x_2 &    +s_1 &                 &  =4              \\
              &    &    x_1      &    +3x_2 &     &   -s_2  &             = 20          \\
              &    &    x_1      &    + x_2 &     &             &             = 10          \\
              &    &    x_1,      &    x_2, &    s_1, &    s_2  &              \ge 0\\ 
  \end{array}
\end{equation*}

Add artificial vriables $y_1, y_2$. The phase I LP is as follows.
\begin{equation*}
  \begin{array}{ccccccccc}
    \text{max}& t= &          &   & &   & -y_1 & -y_2 &        \\ 
    s.t       &    &    1/2x_1      &   + 1/4x_2 &    +s_1 &           &&      &  =4              \\
              &    &    x_1      &    +3x_2 &     &   -s_2  &  +y_1&&           = 20          \\
              &    &    x_1      &    + x_2 &     &             &  &+y_2&           = 10          \\
              &    &    x_1,      &    x_2, &    s_1, &    s_2,  &  y_1,   & y_2 &         \ge 0\\ 
  \end{array}
\end{equation*}

Convert phase I LP to canonical form.
\begin{equation*}
  \begin{array}{ccccccccc}
    \text{max}& t= &    2x_1      & +4x_2  & & -s_2  & -30 &  &        \\ 
    s.t       &    &    1/2x_1      &   + 1/4x_2 &    +s_1 &           &&      &  =4              \\
              &    &    x_1      &    +3x_2 &     &   -s_2  &  +y_1&&           = 20          \\
              &    &    x_1      &    + x_2 &     &             &  &+y_2&           = 10          \\
              &    &    x_1,      &    x_2, &    s_1, &    s_2,  &   y_1,  & y_2 &         \ge 0\\ 
  \end{array}
\end{equation*}

The initial tableau of phase I LP is as follows
\begin{equation*}
  \begin{array}{ccccccc|c|c}
    \hline
    t &  x_1      &  x_2   &  s_1 & s_2 & y_1 & y_2        &   \text{rhs} & \text{Basic var}  \\ \hline
    1 &    -2      &  -4  &    0       & 1  & 0 & 0 & -30          &  t=-30              \\
    0 &    1/2      &  1/4  &    1       & 0  & 0 & 0 & 4          &  s_1=4              \\
    0 &    1      &  3   &    0       & -1 & 1 & 0 & 20            &  y_1 = 20          \\
    0 &    1      &  1   &   0       & 0    & 0 & 1 & 10        &  y_2 = 10          \\ \hline
  \end{array}
\end{equation*}

First iteration: $x_1$ enters, $s_1$ leaves.
\begin{equation*}
  \begin{array}{ccccccc|c|c}
    \hline
    t &  x_1      &  x_2   &  s_1 & s_2 & y_1 & y_2        &   \text{rhs} & \text{Basic var}  \\ \hline
    1 &    0      &  -3  &    4       & 1  & 0 & 0 & -14          &  t=-14              \\
    0 &    1      &  1/2  &    2       & 0  & 0 & 0 & 8          &  x_1=8              \\
    0 &    0      &  5/2   &    -2       & -1 & 1 & 0 & 12            &  y_1 = 12          \\
    0 &    0      &  1/2   &   -2       & 0    & 0 & 1 & 2        &  y_2 = 2          \\ \hline
  \end{array}
\end{equation*}

Second iteration: $x_2$ enters, $y_2$ leaves.
\begin{equation*}
  \begin{array}{ccccccc|c|c}
    \hline
    t &  x_1      &  x_2   &  s_1 & s_2 & y_1 & y_2        &   \text{rhs} & \text{Basic var}  \\ \hline
    1 &    0      &  0  &    -8       & 1  & 0 & 6 & -2          &  t=-2              \\
    0 &    1      &  0  &    4       & 0  & 0 & -1 & 6          &  x_1=6              \\
    0 &    0      &  0   &    8       & -1 & 1 & -5 & 2            &  y_1 = 2          \\
    0 &    0      &  1   &   -4       & 0    & 0 & 2 & 4        &  x_2 = 4          \\ \hline
  \end{array}
\end{equation*}

Third iteration: $s_1$ enters, $y_1$ leaves.
\begin{equation*}
  \begin{array}{ccccccc|c|c}
    \hline
    t &  x_1      &  x_2   &  s_1 & s_2 & y_1 & y_2        &   \text{rhs} & \text{Basic var}  \\ \hline
    1 &    0      &  0  &    0       & 0  & 1 & 1 & 0          &  t=0              \\
    0 &    1      &  0  &    0       & 1/2  & -1/2 & 3/2 & 5          &  x_1=5              \\
    0 &    0      &  0   &    1       & -1/8 & 1/8 & -5/8 & 1/4            &  s_1 = 1/4          \\
    0 &    0      &  1   &  0      & -1/2    & 1/2 & -1/2 & 5        &  x_2 = 5          \\ \hline
  \end{array}
\end{equation*}

The optimal $t=0$. Case 2.1. Obtain a simplex tableau for the original LP.
\begin{equation*}
  \begin{array}{ccccc|c|c}
    \hline
    z &  x_1      &  x_2   &  s_1 & s_2 &   \text{rhs} & \text{Basic var}  \\ \hline
    1 &    0      &  0  &    0       & -1/2  &  25          &  z=25              \\
    0 &    1      &  0  &    0       & 1/2  &  5          &  x_1=5              \\
    0 &    0      &  0   &    1      & -1/8 & 1/4            &  s_1 = 1/4          \\
    0 &    0      &  1   &  0      & -1/2    & 5        &  x_2 = 5          \\ \hline
  \end{array}
\end{equation*}

The curretn BFS is optimal. The optimal solution of the original LP is $(x_1,x_2) = (5,5)$ and the optimal value is $z=25$.

\vspace{\baselineskip}
\noindent
\textbf{Problem 2.}

\noindent
\textbf{(a)}

This problem is feasible if and only if there exists $k$ such that $b/a_k \ge 0$.
\begin{proof}
If there exists $k$ such that $b/a_k \ge 0$, then \{$ x_i = 0 \; (i \ne k)$, $x_k = b/a_k$\} is a feasible solution.

If there doesn't exist $k$ such that $b/a_k \ge 0$, then for all $a_i \ne 0$, $b/a_i < 0$. Suppose $b>0$, then all $a_i \le 0$. Then $\sum_{i=1}^n a_ix_i \le 0 $, thus this problem is not feasible. If $b<0$, then $\sum_{i=1}^n a_ix_i \ge 0 $, and this problem is not feasible.
\end{proof}

\noindent
\textbf{(b)}

Let $K=\{i \mid b/a_i \ge 0\}$.
The set of BFS's is $\{x_i = 0\;(i \ne k), x_k = b/a_k  \mid k\in K\}$. Thus the set of objective function values at BFS's is $\{c_k b / a_k  \mid k\in K\}$.

 Find \[ p = \min_{i\in K} c_i b / a_i \] The optimal solution is $x_i = 0 \; (i\ne p), x_p = b/a_p$ and the optimal value is $c_pb/a_p$.

\vspace{\baselineskip}
\noindent
\textbf{Problem 3.}

Let $z$ be the BFS at a certain step. In each step, replace a basis of $z$ that is not a basis of $y$ by a basis of $y$ that is not a basis of $z$. In this way, the number of shared bases of $y$ and $z$ increases by 1 in each iteration. Thus we can go from $x$ to $y$ in a finite number of steps.

\vspace{\baselineskip}
\noindent
\textbf{Problem 4.}

\noindent
Proposition 1. When formulating the phase I LP, if a non-artificial variable $x$ satisfies:
\[\text{$x$ has coefficient 1 in an equation and zero coefficients in all the other equations. (condition 1)}\]
then we do not need to add an artificial variable in the equation where $x$ has coefficient 1.

When an artificial variable $y$ becomes nonbasic, a non-artificial variable becomes basic and satisfies condition 1. By proposition 1, we can formulate a new phase I LP without $y$. The new phase I LP is equivalent to the LP obtained by eliminating the column of $y$ from the old phase I LP.

\vspace{\baselineskip}
\noindent
\textbf{Problem 5.}

Use Matlab to solve the problems.

\noindent
\textbf{(a)} The problem is unbounded.

\noindent
\textbf{(b)} The optimal solution is $(0, 5, 0, 1, 0, 4)$ and the optimal value is $-74$.


Matlab code for revised simplex method:
\begin{lstlisting}
%% LP (a)
A=[ 0 1 2 1 1 0 -5;
    0 2 1 0 -2 1 0;
    1 1 -2 0 1 0 3];
c = -[2;3;-4;3;1;-4;6];
b = [2;1;3];
B = [4 6 1];

%% LP (b)
A = [1 0 0 -1 1 1;
     1 1 0 -1 3 0;
     1 1 1 -3 0 1];
c = -[14;-19;0;21;52;0];
b = [3;4;6];
B = [1 2 6];

%% revised simplex method
k=0;
while(1)
    N = setdiff(1:length(c), B);
    k=k+1;
    k
    AB = A(:,B);
    AB_inv = inv(AB);

    reduced_cost = -c.' + c(B).'*AB_inv*A
    coef_x = AB_inv*A
    rhs_z=c(B).'*AB_inv*b
    rhs_x=AB_inv*b

    t = find(reduced_cost(N)<0);
    if length(t) == 0
        fprintf('found optimal value\n')
        break
    end
    pivot_column = N(t(1));
    u = AB_inv*A(:,pivot_column);
    t = find(u>0);
    if length(t) == 0
        fprintf('unbounded\n')
        break
    end
    [~,min_i] = min(rhs_x(t) ./ u(t));
    pivot_row = t(min_i);
    B(pivot_row) = pivot_column;
end
z=c(B).'*AB_inv*b;
\end{lstlisting}

% --------------------------------------------------------------
%     You don't have to mess with anything below this line.
% --------------------------------------------------------------

\end{document}