% --------------------------------------------------------------
% This is all preamble stuff that you don't have to worry about.
% Head down to where it says "Start here"
% --------------------------------------------------------------
 
\documentclass[12pt]{article}
 
\usepackage[margin=1in]{geometry} 
\usepackage{amsmath,amsthm,amssymb}
\usepackage{graphicx}

\newcommand{\N}{\mathbb{N}}
\newcommand{\Z}{\mathbb{Z}}
 
\newenvironment{theorem}[2][Theorem]{\begin{trivlist}
\item[\hskip \labelsep {\bfseries #1}\hskip \labelsep {\bfseries #2.}]}{\end{trivlist}}
\newenvironment{lemma}[2][Lemma]{\begin{trivlist}
\item[\hskip \labelsep {\bfseries #1}\hskip \labelsep {\bfseries #2.}]}{\end{trivlist}}
\newenvironment{exercise}[2][Exercise]{\begin{trivlist}
\item[\hskip \labelsep {\bfseries #1}\hskip \labelsep {\bfseries #2.}]}{\end{trivlist}}
\newenvironment{reflection}[2][Reflection]{\begin{trivlist}
\item[\hskip \labelsep {\bfseries #1}\hskip \labelsep {\bfseries #2.}]}{\end{trivlist}}
\newenvironment{proposition}[2][Proposition]{\begin{trivlist}
\item[\hskip \labelsep {\bfseries #1}\hskip \labelsep {\bfseries #2.}]}{\end{trivlist}}
\newenvironment{corollary}[2][Corollary]{\begin{trivlist}
\item[\hskip \labelsep {\bfseries #1}\hskip \labelsep {\bfseries #2.}]}{\end{trivlist}}

\def\name{Zhenghan Fang}

\usepackage{fancyhdr}
\pagestyle{fancy}
\fancyhf{}
\rhead{\name}
\cfoot{\thepage}
\renewcommand{\headrulewidth}{0pt}

\begin{document}

% --------------------------------------------------------------
%                         Start here
% --------------------------------------------------------------
 
%\renewcommand{\qedsymbol}{\filledbox}
 
\title{STOR 614 - Linear Programming, Spring 2019 \\
Homework No. 2}
\author{\name}

\maketitle

\noindent
\textbf{Problem 1.}

Suppose that $P$ contains a line $Q = \left\{ x+ \lambda d \mid \lambda \in \mathbb{R} \right\}$, where $x\in \mathbb{R}, d\in \mathbb{R}, d \ne 0$. Because the set $\left\{ a_1, \hdots, a_m \right\}$ contains $n$ linearly independent vectors, there exists $a_k$, such that $a_k^T d \ne 0$. Therefore,
$$\left\{ a_k^T q \mid q \in Q \right\} = \left\{ a_k^T x + \lambda a_k^T d \mid \lambda \in \mathbb{R} \right\} = \mathbb{R} , $$ which contradicts with $$a_k^T q \ge b_k, \text{\;for all\;} q\in Q \quad$$

\vspace{\baselineskip}
\noindent
\textbf{Problem 2.}

Basic feasible solutions:
\begin{gather*}
    \begin{bmatrix} x_1 \\ x_2 \end{bmatrix} = 
    \begin{bmatrix} 1/3 \\ 2/3 \end{bmatrix}, 
    \begin{bmatrix} 0 \\ 1 \end{bmatrix},
    \begin{bmatrix} 1/3 \\ 2/3 \end{bmatrix},
    \begin{bmatrix} 1/2 \\ 0 \end{bmatrix},
    \begin{bmatrix} 0 \\ 0 \end{bmatrix}.
\end{gather*}

Degenerate BFS:
\begin{align*}
    \begin{bmatrix} x_1 \\ x_2 \end{bmatrix} = 
    \begin{bmatrix} 1/3 \\ 2/3 \end{bmatrix}
\end{align*}
because it has 3 active constraints:
\begin{gather*}
    x_1+x_2 \le 1 \\ 4x_1 + x_2 \le 2 \\ 2x_1 +x_2 \le 4/3 .
\end{gather*}

\vspace{\baselineskip}
\noindent
\textbf{Problem 3.}
\begin{alignat*}{7}
    &\text{min}\quad      & z = \;& 3x_1 &&& + x_2 & \\
    &\text{s.t}  & & x_1 &&& && - \beta_1 && && = 3 \\
    & & & x_1  &&& +x_2 &&  && +\beta_2 &&= 4, \\
    & & & 2x_1  &&& - x_2 && && &&= 3, \\
    & & & x_1, &&& x_2, &&\beta_1, &&\beta_2  &&\ge 0
\end{alignat*}

\vspace{\baselineskip}
\noindent
\textbf{Problem 4.}

\noindent
\textbf{Problem 4(a)} 

True.

Let $A \in \mathbb{R}^{m\times n}$. Because $A$ has full row rank, $m \le n$. Let $x_{B1}$ and $x_{B2}$ be the bases of $x$. Let $x_{N1}$ and $x_{N2}$ be the corresponding collections of nonbasic variables, i.e.
\begin{gather*}
    x_{N1} = \{x_1, \hdots, x_n\} \setminus x_{B1} \\
    x_{N2} = \{x_1, \hdots, x_n\} \setminus x_{B2}
\end{gather*}
Let $n_0$ be the number of zeros in $x$. Then,
\begin{align*}
    & x_{B1} \ne x_{B2} \\
    \Rightarrow \quad & x_{N1} \ne x_{N2} \\
    \Rightarrow \quad & \left| x_{N1} \cup x_{N2} \right| > n-m \quad \text{($\left|\cdot\right|$ denotes cardinality)}\\
    \Rightarrow \quad & n_0 > n-m
\end{align*}
Therefore, $x$ is degenerate.

\vspace{\baselineskip}
\noindent
\textbf{Problem 4(b)} 

False.

Counter example:
$A=\begin{bmatrix} 1 & 0 \end{bmatrix}$, $b=\begin{bmatrix} 0 \end{bmatrix}$. $x=\begin{bmatrix} 0 & 0 \end{bmatrix}^T$ is a degenerate basic solution but has only one basis, $\begin{bmatrix} 1 \end{bmatrix}$.

\vspace{\baselineskip}
\noindent
\textbf{Problem 5.}

\noindent
\textbf{Problem 5(a)} 
\begin{proposition}{1}
Let $A \in \mathbb{R}^{m\times n}$, $b\in \mathbb{R}^n, b \ne 0$. If 
\begin{enumerate}
    \item $Ax=b$ has a solution $x^*$, $x^*_{i} > 0$, for $i=1, \hdots, n$
    \item The columns of $A$ are not linearly independent
\end{enumerate}
then there exists a solution of $Ax=b$, $x'$, such that
\begin{enumerate}
    \item $x'_i \ge 0$, for $i=1, \hdots, n$, 
    \item The columns of $A$ corresponding to nonzero entries of $x'$ are linearly independent.
\end{enumerate} 
\end{proposition}

\begin{proof}
Assume the induction hypothesis that the proposition holds for all $n < N (N \ge 3)$. Consider $n=N$. Let $d$ be a nontrivial solution of $Ax=0$. Then, $\exists \lambda \in \mathbb{R}$, such that
$
    y = (x^* + \lambda d )
$
has at least one zero entry and no negative entries. Let $A' \in \mathbb{R}^{m\times N'}$ be the matrix containing the columns of $A$ corresponding to nonzero entries of $y$ ($N'<N$). If the columns of $A'$ are linearly independent, then $x'=y$. Otherwise, by induction hypothesis, there exists a solution of $A'x=b$, $y'$, such that $y'_i \ge 0$, for $i=1, \hdots, N'$, and the columns of $A'$ corresponding to nonzero entries of $y'$ are linearly independent. Then, $x'$ can be obtained by replacing the nonzero entries of $y$ by corresponding entries of $y'$.
\end{proof}

\noindent
{\em Proof for problem 5(a).}

If $b=0$, then $0$ is a degenerate basic feasible solution of $P$, so $b\ne 0$.

Suppose that $x$ is not a basic feasible solution, then the columns of $A$ corresponding to nonzero entries of $x$ are not linearly independent. By proposition 1, there exists $x' \in \mathbb{R}^{n}$, such that 
$
x'$ is a feasible solution, $x'$ has more than $(n-m)$ zeros, and the columns of $A$ corresponding to nonzero entries of $x'$ are linearly independent. Therefore, $x'$ is a degenerate basic feasible solution, which contradicts with that all basic feasible solutions are nondegenerate.


\vspace{\baselineskip}
\noindent
\textbf{Problem 5(b)}
A counter example:
\begin{align*}
    A&=\begin{bmatrix} 2&0 \end{bmatrix} \\
    b&=\begin{bmatrix} 0\\0 \end{bmatrix} \\
    x&=\begin{bmatrix} 0\\1 \end{bmatrix}
\end{align*}
$x$ has exactly 1 positive component but is not a basic feasible solution.



% --------------------------------------------------------------
%     You don't have to mess with anything below this line.
% --------------------------------------------------------------

\end{document}