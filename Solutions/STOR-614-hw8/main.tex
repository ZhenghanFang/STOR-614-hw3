% --------------------------------------------------------------
% This is all preamble stuff that you don't have to worry about.
% Head down to where it says "Start here"
% --------------------------------------------------------------
 
\documentclass[12pt]{article}
\usepackage{amsgen,amsmath,amstext,amsbsy,amsopn,amssymb,mathabx,amsthm,bm,bbm,romannum,enumitem}
\usepackage[dvips]{graphicx}

\usepackage[pagebackref,bookmarksnumbered]{hyperref}
\usepackage{url}
\hypersetup{
	colorlinks=true,
	linkcolor=red,
	filecolor=magenta,      
	urlcolor=blue,
}

\setcounter{tocdepth}{3}
\usepackage[depth=3]{bookmark}

\usepackage[margin=1in]{geometry}
\renewcommand{\baselinestretch}{1.5}	% Line Stretch

\usepackage[utf8]{inputenc}

%----- theorems -----%

\newtheorem{thm}{Theorem}[section]
\newtheorem{lem}[thm]{Lemma}
\newtheorem{prop}[thm]{Proposition}
\newtheorem{coro}[thm]{Corollary}

\theoremstyle{definition}
\newtheorem{dfn}{Definition}[section]
\newtheorem*{pchln}{Punchline}

\theoremstyle{remark}
\newtheorem*{rmk}{Remark}
\newtheorem*{note}{Note}
\newtheorem{eg}{Example}[section]
\newtheorem{fact}{Fact}[section]
\newtheorem*{hint}{Hint}


%----- bold fonts -----%

\newcommand{\ab}{\mathbf{a}}
\newcommand{\bbb}{\mathbf{b}}
\newcommand{\cbb}{\mathbf{c}}
\newcommand{\db}{\mathbf{d}}
\newcommand{\eb}{\mathbf{e}}
\newcommand{\fb}{\mathbf{f}}
\newcommand{\gb}{\mathbf{g}}
\newcommand{\hb}{\mathbf{h}}
\newcommand{\ib}{\mathbf{i}}
\newcommand{\jb}{\mathbf{j}}
\newcommand{\kb}{\mathbf{k}}
\newcommand{\lb}{\mathbf{l}}
\newcommand{\mb}{\mathbf{m}}
\newcommand{\nbb}{\mathbf{n}}
\newcommand{\ob}{\mathbf{o}}
\newcommand{\pb}{\mathbf{p}}
\newcommand{\qb}{\mathbf{q}}
\newcommand{\rb}{\mathbf{r}}
\newcommand{\sbb}{\mathbf{s}}
\newcommand{\tb}{\mathbf{t}}
\newcommand{\ub}{\mathbf{u}}
\newcommand{\vb}{\mathbf{v}}
\newcommand{\wb}{\mathbf{w}}
\newcommand{\xb}{\mathbf{x}}
\newcommand{\yb}{\mathbf{y}}
\newcommand{\zb}{\mathbf{z}}

% denote vectors
\newcommand{\ba}{\bm{a}}
\newcommand{\bb}{\bm{b}}
\newcommand{\bc}{\bm{c}}
\newcommand{\bd}{\bm{d}}
\newcommand{\be}{\bm{e}}
\newcommand{\bbf}{\bm{f}}
\newcommand{\bg}{\bm{g}}
\newcommand{\bh}{\bm{h}}
\newcommand{\bi}{\bmf{i}}
\newcommand{\bj}{\bm{j}}
\newcommand{\bk}{\bm{k}}
\newcommand{\bl}{\bm{l}}
\newcommand{\bbm}{\bm{m}}
\newcommand{\bn}{\bm{n}}
\newcommand{\bo}{\bm{o}}
\newcommand{\bp}{\bm{p}}
\newcommand{\bq}{\bm{q}}
\newcommand{\br}{\bm{r}}
\newcommand{\bs}{\bm{s}}
\newcommand{\bt}{\bm{t}}
\newcommand{\bu}{\bm{u}}
\newcommand{\bv}{\bm{v}}
\newcommand{\bw}{\bm{w}}
\newcommand{\bx}{\bm{x}}
\newcommand{\by}{\bm{y}}
\newcommand{\bz}{\bm{z}}

% denote random matrices
\newcommand{\Ab}{\mathbf{A}}
\newcommand{\Bb}{\mathbf{B}}
\newcommand{\Cb}{\mathbf{C}}
\newcommand{\Db}{\mathbf{D}}
\newcommand{\Eb}{\mathbf{E}}
\newcommand{\Fb}{\mathbf{F}}
\newcommand{\Gb}{\mathbf{G}}
\newcommand{\Hb}{\mathbf{H}}
\newcommand{\Ib}{\mathbf{I}}
\newcommand{\Jb}{\mathbf{J}}
\newcommand{\Kb}{\mathbf{K}}
\newcommand{\Lb}{\mathbf{L}}
\newcommand{\Mb}{\mathbf{M}}
\newcommand{\Nb}{\mathbf{N}}
\newcommand{\Ob}{\mathbf{O}}
\newcommand{\Pb}{\mathbf{P}}
\newcommand{\Qb}{\mathbf{Q}}
\newcommand{\Rb}{\mathbf{R}}
\newcommand{\Sbb}{\mathbf{S}}
\newcommand{\Tb}{\mathbf{T}}
\newcommand{\Ub}{\mathbf{U}}
\newcommand{\Vb}{\mathbf{V}}
\newcommand{\Wb}{\mathbf{W}}
\newcommand{\Xb}{\mathbf{X}}
\newcommand{\Yb}{\mathbf{Y}}
\newcommand{\Zb}{\mathbf{Z}}

% denote random vectors
\newcommand{\bA}{\bm{A}}
\newcommand{\bB}{\bm{B}}
\newcommand{\bC}{\bm{C}}
\newcommand{\bD}{\bm{D}}
\newcommand{\bE}{\bm{E}}
\newcommand{\bF}{\bm{F}}
\newcommand{\bG}{\bm{G}}
\newcommand{\bH}{\bm{H}}
\newcommand{\bI}{\bm{I}}
\newcommand{\bJ}{\bm{J}}
\newcommand{\bK}{\bm{K}}
\newcommand{\bL}{\bm{L}}
\newcommand{\bM}{\bm{M}}
\newcommand{\bN}{\bm{N}}
\newcommand{\bO}{\bm{O}}
\newcommand{\bP}{\bm{P}}
\newcommand{\bQ}{\bm{Q}}
\newcommand{\bR}{\bm{R}}
\newcommand{\bS}{\bm{S}}
\newcommand{\bT}{\bm{T}}
\newcommand{\bU}{\bm{U}}
\newcommand{\bV}{\bm{V}}
\newcommand{\bW}{\bm{W}}
\newcommand{\bX}{\bm{X}}
\newcommand{\bY}{\bm{Y}}
\newcommand{\bZ}{\bm{Z}}

% denote vectors
\newcommand{\bbeta}{\bm{\beta}}
\newcommand{\balpha}{\bm{\alpha}}
\newcommand{\bgamma}{\bm{\gamma}}
\newcommand{\blambda}{\bm{\lambda}}
\newcommand{\bomega}{\bm{\omega}}
\newcommand{\bmu}{\bm{\mu}}
\newcommand{\bepsilon}{\bm{\epsilon}}
\newcommand{\btheta}{\bm{\theta}}
\newcommand{\bphi}{\bm{\phi}}
\newcommand{\bvarphi}{\bm{\varphi}}
\newcommand{\bxi}{\bm{\xi}}
\newcommand{\bpi}{\bm{\pi}}

% denote matrices
\newcommand{\bGamma}{\bm{\Gamma}}
\newcommand{\bLambda}{\bm{\Lambda}}
\newcommand{\bSigma}{\bm{\Sigma}}

% others
\newcommand{\bcE}{\bm{\mathcal{E}}}	% filtration
\newcommand{\bcF}{\bm{\mathcal{F}}}	% filtration
\newcommand{\bcG}{\bm{\mathcal{G}}}	% filtration


%----- double fonts -----%

\newcommand{\bbR}{\mathbb{R}}
\newcommand{\bbE}{\mathbb{E}}
\newcommand{\bbN}{\mathbb{N}}
\newcommand{\bbP}{\mathbb{P}}
\newcommand{\bbQ}{\mathbb{Q}}
\newcommand{\bbZ}{\mathbb{Z}}


%----- script fonts -----%

\newcommand{\cA}{\mathcal{A}}
\newcommand{\cB}{\mathcal{B}}
\newcommand{\cC}{\mathcal{C}}
\newcommand{\cD}{\mathcal{D}}
\newcommand{\cE}{\mathcal{E}}
\newcommand{\cF}{\mathcal{F}}
\newcommand{\cG}{\mathcal{G}}
\newcommand{\cH}{\mathcal{H}}
\newcommand{\cI}{\mathcal{I}}
\newcommand{\cJ}{\mathcal{J}}
\newcommand{\cK}{\mathcal{K}}
\newcommand{\cL}{\mathcal{L}}
\newcommand{\cM}{\mathcal{M}}
\newcommand{\cN}{\mathcal{N}}
\newcommand{\cO}{\mathcal{O}}
\newcommand{\cP}{\mathcal{P}}
\newcommand{\cQ}{\mathcal{Q}}
\newcommand{\cR}{\mathcal{R}}
\newcommand{\cS}{\mathcal{S}}
\newcommand{\cT}{\mathcal{T}}
\newcommand{\cU}{\mathcal{U}}
\newcommand{\cV}{\mathcal{V}}
\newcommand{\cW}{\mathcal{W}}
\newcommand{\cX}{\mathcal{X}}
\newcommand{\cY}{\mathcal{Y}}
\newcommand{\cZ}{\mathcal{Z}}


%----- special operators -----%

\newcommand{\argmin}{\mathop{\mathrm{argmin}}}
\newcommand{\argmax}{\mathop{\mathrm{argmax}}}

\newcommand{\bvar}{\textbf{Var}}
\newcommand{\bbias}{\textbf{Bias}}
\newcommand{\bcov}{\textbf{Cov}}
\newcommand{\bcor}{\textbf{Cor}}
\newcommand{\brank}{\textbf{rank}}
\newcommand{\bsign}{\textbf{sign}}
\newcommand{\bdiag}{\textbf{diag}}	% diagonal
\newcommand{\bdim}{\textbf{dim}}	% dimension
\newcommand{\btr}{\textbf{tr}}	    % trace
\newcommand{\bspan}{\textbf{span}}	% linear span
\newcommand{\bsupp}{\textbf{supp}}	% support
\newcommand{\bepi}{\textbf{epi}}	% epigraph

\newcommand{\perm}{\textbf{Perm}}	% permutation

\newcommand{\wass}{\textbf{Wass}}	% Wasserstein Distance
\newcommand{\ks}{\textbf{KS}}		% Kolomogov-Smirnov Distance

\newcommand{\brem}{\textbf{Rem}}		% remainders


\newcommand{\bzero}{{\mathbf{0}}}	% zero vector
\newcommand{\bone}{{\mathbf{1}}}	% all-one vector
\newcommand{\bbone}{{\mathbbm{1}}}	% indicator

\newcommand{\rmd}{\mathrm{d}}		% differentiation

\newcommand\indep{\protect\mathpalette{\protect\independenT}{\perp}}
\def\independenT#1#2{\mathrel{\rlap{$#1#2$}\mkern2mu{#1#2}}}	% independence

%----- distribution name -----%

\newcommand{\Exp}{\textbf{Exp}}
\newcommand{\Pois}{\textbf{Pois}}
\newcommand{\Gumb}{\textbf{Gumbel}}
\newcommand{\Bern}{\textbf{Bernoulli}}
\newcommand{\Bin}{\textbf{Bin}}
\newcommand{\NBin}{\textbf{NBin}}
\newcommand{\Multi}{\textbf{Multi}}
\newcommand{\Geo}{\textbf{Geo}}
\newcommand{\Hyper}{\textbf{Hyper}}
\newcommand{\SBM}{\textbf{SBM}}
\newcommand{\PoisProc}{\textbf{PoisProc}}

\usepackage{titling}

% Create subtitle command for use in maketitle
\newcommand{\subtitle}[1]{
	\posttitle{
		\begin{center}\large#1\end{center}
	}
}
 
\usepackage[margin=1in]{geometry} 
\usepackage{amsmath,amsthm,amssymb}
\usepackage{graphicx}
\usepackage{float}
\usepackage{enumerate,enumitem}

\newcommand{\N}{\mathbb{N}}
\newcommand{\Z}{\mathbb{Z}}
 
\newenvironment{theorem}[2][Theorem]{\begin{trivlist}
\item[\hskip \labelsep {\bfseries #1}\hskip \labelsep {\bfseries #2.}]}{\end{trivlist}}
\newenvironment{lemma}[2][Lemma]{\begin{trivlist}
\item[\hskip \labelsep {\bfseries #1}\hskip \labelsep {\bfseries #2.}]}{\end{trivlist}}
\newenvironment{exercise}[2][Exercise]{\begin{trivlist}
\item[\hskip \labelsep {\bfseries #1}\hskip \labelsep {\bfseries #2.}]}{\end{trivlist}}
\newenvironment{reflection}[2][Reflection]{\begin{trivlist}
\item[\hskip \labelsep {\bfseries #1}\hskip \labelsep {\bfseries #2.}]}{\end{trivlist}}
\newenvironment{proposition}[2][Proposition]{\begin{trivlist}
\item[\hskip \labelsep {\bfseries #1}\hskip \labelsep {\bfseries #2.}]}{\end{trivlist}}
\newenvironment{corollary}[2][Corollary]{\begin{trivlist}
\item[\hskip \labelsep {\bfseries #1}\hskip \labelsep {\bfseries #2.}]}{\end{trivlist}}

\usepackage[framed,numbered,autolinebreaks,useliterate]{mcode}

\def\name{Zhenghan Fang}

\usepackage{fancyhdr}
\pagestyle{fancy}
\fancyhf{}
\rhead{\name}
\cfoot{\thepage}
\renewcommand{\headrulewidth}{0pt}

\begin{document}

% --------------------------------------------------------------
%                         Start here
% --------------------------------------------------------------
 
%\renewcommand{\qedsymbol}{\filledbox}

\pagenumbering{arabic}

\title{STOR 614 - Linear Programming, Spring 2019 \\
Homework No. 8}
\author{\name}

\maketitle

\noindent
\textbf{Problem 1.}

(1) \[ Ad^* = -A_B A_B^{-1} A_j + A_j = 0. \]

(2) \[c^Td^* = -c_B^T A_B^{-1} A_j + c_j = -(c_B^T A_B^{-1} A_j - c_j) >0 \]
because the reduced cost of $x_j$ is $c_B^T A_B^{-1} A_j - c_j$ and is negative.

(3) The matrix
\[\begin{bmatrix} A_B & D \\ 0 & I_{n-m-1} \end{bmatrix} \in \mathbb{R}^{(n-1)\times(n-1)}\]
is nonsingular because $A_B$ is nonsingular. Thus, the matrix with the active constraints of $d^*$ as row vectors has $n-1$ linearly independent columns. Thus, $d^*$ has $n-1$ linearly independent active constraints.

\noindent
\textbf{Problem 2.}

For any $x, y\in \mathbb{R}^n$ and $0 \le t \le 1$,
\begin{align*}
    F[(1-t) x+ty] & = g\{f[(1-t) x+ty]\} \\
    & \le g[(1-t)f(x)+tf(y)] \quad \text{($f$ is convex and $g$ is nondecreasing)} \\
    & \le (1-t)g(f(x)) + tg(f(y)) \quad \text{($g$ is convex)}\\
    & = (1-t)F(x) + tF(y)
\end{align*}
Thus, $F$ is convex.

\noindent
\textbf{Problem 3.}

\noindent
\textbf{(a)}

First, we have
\[M = \begin{bmatrix} 2&0\\0&8 \end{bmatrix}, c= \begin{bmatrix} -8\\-16 \end{bmatrix}, A=\begin{bmatrix} -1&-1\\-1&0\\1&0\\0&1 \end{bmatrix},b=\begin{bmatrix} -5\\-3\\0\\0 \end{bmatrix}.\]

The matrix $M$ is positive definite.

The point $(3,2)$ satisfies the first two constraints as equalities and the last two strictly. So the multipliers would need to satisfy $u_1\ge 0,u_2\ge0,u_3=0,u_4=0$. The equation $Mx+c=A^Tu$ becomes 
\[\begin{bmatrix}-2\\0\end{bmatrix} = \begin{bmatrix} -1&-1&1&0\\-1&0&0&1 \end{bmatrix} \begin{bmatrix} u_1\\u_2\\0\\0 \end{bmatrix}\]
which gives $u_1=0$ and $u_2=2$. Therefore there exists $u\in \mathbb{R}^4$ such that $(x,u)$ satisfies all the KKT conditions. $x = (3, 2)$ is a global
solution.

$M$ is positive definite, thus $z$ is strictly convex, thus the QP has a unique global solution.


\noindent
\textbf{(b)}

First, we have 
\[M = \begin{bmatrix} 1&-1\\-1&2 \end{bmatrix}, c= \begin{bmatrix} -2\\-6 \end{bmatrix}, A=\begin{bmatrix} -1&-1\\1&-2\\-2&-1\\1&0\\0&1 \end{bmatrix},b=\begin{bmatrix} -2\\-2\\-3\\0\\0 \end{bmatrix}.\]

The matrix $M$ is positive definite. 
The point $(2/3,4/3)$ satisfies the first two constraints as equalities and the last three strictly. So the multipliers would need to satisfy $u_1\ge0,u_2\ge0,u_3=0,u_4=0,u_5=0$. The equation $Mx+c=A^Tu$ becomes 
\[\begin{bmatrix}-8/3\\-4\end{bmatrix} = \begin{bmatrix} -1&1&-2&1&0\\-1&-2&-1&0&1 \end{bmatrix} \begin{bmatrix} u_1\\u_2\\0\\0\\0 \end{bmatrix}\]
which gives $u_1=28/9,u_2=4/9$. Therefore there exists $u\in \mathbb{R}^5$ such that $(x,u)$ satisfies all the KKT conditions. $x = (2/3, 4/3)$ is a global solution.

$M$ is positive definite, thus $z$ is strictly convex, thus the QP has a unique global solution.

\noindent
\textbf{Problem 4.}

\begin{align*}
    \text{min}\quad & \frac{1}{2}x^Tx \\
    \text{s.t.}\quad & a^Tx+\alpha \ge 0
\end{align*}

We have \[M=I_n, c=0, A=a^T, b=-\alpha\]. The KKT conditions state that
\begin{align*}
\begin{cases}
    I_nx=au \\
    a^Tx+\alpha \ge 0 \\
    u \ge 0 \\
    (a^Tx+\alpha)u=0
\end{cases}
\end{align*}

If $\alpha \ge 0$, then $u=0, x=0$. The optimal solution is $x=0$, and the optimal value is $0$.

If $\alpha < 0$, then $a^Tx+\alpha = a^Tau+\alpha = 0 \implies u=-\alpha/(a^Ta)$ (assume $a\ne 0$). The optimal solution is \[x=au = -\frac{\alpha} {a^Ta}a,\] and the optimal value is \[\frac{1}{2}x^Tx = \frac{1}{2} \alpha^2/(a^Ta).\]


% --------------------------------------------------------------
%     You don't have to mess with anything below this line.
% --------------------------------------------------------------

\end{document}